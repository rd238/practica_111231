\documentclass[12pt, a4paper]{article}

\usepackage[english, russian]{babel}
\usepackage[T2A]{fontenc}
\usepackage[utf8]{inputenc}

\usepackage{amsmath}
\usepackage{tikz}
\usepackage{verbatim}

\usepackage[left=3cm,right=1.5cm,top=2cm,bottom=2cm,bindingoffset=0cm]{geometry}

\pagestyle{empty}
\linespread{1.5}






\begin{document}
	
	\begin{center}
	 	{\large МИНИСТЕРСТВО НАУКИ И ВЫСШЕГО ОБРАЗОВАНИЯ РОССИЙСКОЙ ФЕДЕРАЦИИ} 

		{\large ФЕДЕРАЛЬНОЕ ГОССУДАРСТВЕННОЕ БЮДЖЕТНОЕ ОБРАЗОВАТЕЛЬНОЕ УЧРЕЖДЕНИЕ ВЫСШЕГО ОБРАЗОВАНИЯ}
		\begin{center}
			{\large \textbf{БЕЛГОРОДСКИЙ ГОССУДАРСТВЕННЫЙ ТЕХНОЛОГИЧЕСКИЙ УНИВЕРСИТЕТ \mbox{им. В. Г. Шухова} (БГТУ им. В. Г. Шухова)}}
		\end{center}	
	\end{center}
	
	\begin{center}
		{\large ИНСТИТУТ ИНФОРМАЦИОННЫХ ТЕХНОЛОГИЙ И УПРАВЛЯЮЩИХ СИСТЕМ}
	\end{center}
	\begin{center}
		{\large Кафедра программного обеспечения вычислительной техники и автоматизированных систем}
	\end{center}
	\begin{center}
		\vspace{1cm}
		{\textbf {\Large Отчет}}

		{\Large По учебно-ознакомительной практике}
	
	\end{center}

	\begin{flushright}
		\vspace{1cm}
		{\large Выполнил: студент группы КБ-232}

		{\large Башков Михаил Антонович}

		\vspace{0.5cm}
		\begin{tikzpicture}
			\draw[black, thick] (0,0) -- (5,0);
		\end{tikzpicture}

		{(подпись студента)}

		\vspace{0.5cm}
		{\large Проверил: асистент \mbox{Новожен Н.В.}}
		\vspace{0.5cm}

		\begin{tikzpicture}
			\draw[black, thick] (0,0) -- (5,0);
		\end{tikzpicture}

		{(подпись руководителя практики)}

		\vspace{0.3cm}
		{\large Оценка:}
		\begin{tikzpicture}
			\draw[black, thick] (0,0) -- (3,0);
		\end{tikzpicture}

	\end{flushright}
	
	\begin{center}
		\vspace{3cm}
		{\large Белгород 2024 г.}
	\end{center}
	




	\begin{center}
		\newpage
		{\Large \textbf{Оглавление}}
	\end{center}
	\begin{flushleft}
		{\large \textbf{Компьютерная практика}}
		
		{1. Tема 1. Линейные алгоритмы} \par
		{2. Tема 2. Разветвляющиеся алгоритмы} \par
		{3. Tема 3. Циклические и итерационные алгоритмы} \par
		{4. Tема 4. Простейшие операции над массивами} \par
		{5. Tема 5. Векторы и матрицы} \par
		{6. Tема 6. Линейный поиск} \par
		{7. Tема 7. Арифметика} \par
		{8. Tема 8. Геометрия и теория множеств} \par
		{9. Tема 9. Линейная алгебра и сжатие информации} \par
		{10. Tема 10. Алгоритмы обработки символьной информации} \par
		{11. Tема 11. Аналитическая геометрия} \par
		{12. Tема 12. Кривые второго порядка на плоскости} \par
		{13. Tема 13. Графическое решение систем уравнений} \par
		{14. Tема 14. Плоскость в трехмерном пространстве} \par
		{15. Tема 15. Поверхность второго порядка в трехмерном пространстве} \par
	\end{flushleft}



	\begin{flushleft}
		\newpage
		{\Large \textbf{Задания к работе}} \par
		{Tема 1. \par Угол $\alpha$ задан в градусах, минутах и секундах.
		Найти его величину в радианах (с максимально возможной точностью).} \par
		{Tема 2. \par Заданы три числа: a, b, c. Определить, могут ли они быть сторонами
		треугольника, и если да, то определить его тип:
		равносторонний,	равнобедренный, разносторонний.} \par
		{Tема 3. \par Численно убедиться, является ли заданная функция $y=f(x)$
		чётной или нёчетной на заданном отрезке $-a\leq x \leq a$. Учесть погрешность
		вычислений и возможные точки разрыва функции.} \par
		{Tема 4. \par В массиве $C(n)$ подсчитать количество отрицательных и сумму
		положительных элементов.} \par
		{Tема 5. \par Строки матрицы $A(m,n)$ заполнены не полностью: в массиве
		$L(m)$ указано количество элементов в каждой строке. Переслать элементы
		 матрицы построчно в начало одномерного массива $T(m\cdot n)$ и
		подсчитать их количество.} \par
		{Tема 6. \par Седловой точкой в матрице называется элемент, являющийся
		одновременно наибольшим в столбце и наименьшим в строке. Седловых 
		точек может быть несколько. В матрице $A(m,n)$ найти все седловые
		точки либо установить, что таких точек нет.} \par
		{Tема 7. \par Натуральное число в $p$-ичной системе счисления задано своими
		цифрами, хранящимися в массиве $K(n)$. Проверить корректность такого 
		представления и перевести число в $q$-ичную систему (возможно,
		число слишком велико, чтобы получить его внутреннее представление;
		кроме того, $p\leq 10$, $q\leq 10$).} \par
		{Tема 8. \par Заяц, хаотично прыгая, оставил след в виде замкнутой самопересекающейся
		ломаной, охватывающей территорию его владения (отрезки
		ломаной заданы длиной прыжка и его направлением по азимуту).
		Найти площадь минимального по площади выпуклого многоугольника,
		описанного вокруг этой территории.} \par
		\newpage
		{Tема 9. \par Выполнить операцию транспонирования прямоугольной матрицы
		$A(m,n)$, $m\neq n$, не выделяя дополнительного массива для хранения
		результата. Матрицу представить в виде одномерного массива.} \par
		{Tема 10. \par Текст записан одной длинной строкой. Признаком красной строки
		служит символ \$. Переформатировать текст в 60-символьные строки,
		формируя абзацы.} \par
		{Tема 11. \par Построить прямую параллельную оси абсцисс $(Ox)$ и пересекающую
		ось ординат $(Oy)$ в точке $A(0,2)$ в диапазоне $x\in [-3;3]$ с шагом $\Delta =0.5$.} \par
		{Tема 12. \par Построить верхнюю часть параболы $y^2=x$ при $0\leq x \leq 4$ с шагом $\Delta =0.25$.} \par
		{Tема 13. \par 
		\begin{equation*} 
			\begin{cases} 
				y=\frac{2}{x}+2
				\\
				z=x^2+1
			\end{cases} 
		\end{equation*}
		\begin{center}в диапазоне $0.2\leq x\leq 3$, с шагом $\Delta 0.1$.\end{center}} \par
		{Tема 14. \par Построить плоскость, параллельную плоскости $Oxy$ и пересекающую ось $Oz$
		в точке $M(0,0,2)$, при $0\leq x\leq 6$ с шагом $\Delta =0.5$ и $0\leq y\leq 6$
		с шагом $\Delta =1$.} \par
		{Tема 15. \par Построить верхнюю часть эллипсоида, заданного уравнением 
		$\frac{x^2}{9}+\frac{y^2}{4}+z^2=1$, лежащую в диапазоне $-3\leq x\leq 3$, $-2\leq y\leq 2$ с 
		шагом $\Delta =0.5$ для обеих переменных.} \par
	\end{flushleft}





	\begin{center}
		\newpage
		{\Large \textbf{Основная часть (выполнение заданий)}}
	\end{center}
	\begin{flushleft}
		\vspace{0.5cm}{\textbf{Задание 1}}\par
		{\textbf{Tема 1. Линейные алгоритмы}}\par
		{Угол $\alpha$ задан в градусах, минутах и секундах.
		Найти его величину в радианах (с максимально возможной точностью)}\par
		{\textbf{Математическое обоснование и словесное описание}\par
		Чтобы перевести угол, заданный в градусах, минутах и секундах в радианы,
		сначала необходимо градусы, минуты и секунды перевести в градусы.
		$degrees^\circ = degrees^\circ +\frac{minutes}{60}+\frac{second}{3600}$
		А потом переводим градусы в радианы руководствуясь данной формулой
		$R=\frac{degrees^\circ\cdot \pi}{180}$}\par

		{\textbf{Блок-схема}\par 
		\begin{center}
			\begin{tikzpicture}
				\filldraw [fill=black!5, very thick] ellipse (1.5 and 0.5);
				\node at (0, 0) {Начало};
				\draw [thick, ->] (0,-0.5) -- (0, -1);
				\draw (-2,-1)--(2,-1)--(2.5,-1.7)--(-1.5,-1.7)--cycle;
				\node at (0.2,-1.4) {ввод grad, min, sec};
				\draw [thick, ->] (0,-1.7) -- (0, -2.2);
				\draw (-3.5,-2.2)--(3.5, -2.2)--(3.5,-2.9)--(-3.5,-2.9)--cycle;
				\node at (0, -2.5) {grad:=grad+(min+sec div 60) div 60};
				\draw [thick, ->] (0,-2.9) -- (0, -3.4);
				\draw (-2,-3.4)--(2, -3.4)--(2,-4)--(-2,-4)--cycle;
				\node at (0, -3.7) {R:=grad $\cdot\pi$ div 180};
				\draw [thick, ->] (0,-4) -- (0, -4.5);
				\draw (-2,-4.5)--(2,-4.5)--(2.5,-5.1)--(-1.5,-5.1)--cycle;
				\node at (0,-4.8) {вывод R};
				\draw [thick, ->] (0,-5.1) -- (0, -5.6);
				\filldraw [fill=black!5, very thick] (0, -6.1) ellipse (1.5 and 0.5);
				\node at (0,-6.1) {Конец};
			\end{tikzpicture}\par
		\end{center}}

		{\textbf{Код программы}}\par
		\begin{verbatim}
			double zadanie1(double grad, double min, double sec){
    			return (grad + (min + sec / 60) / 60) * 3.14 / 180;
			}
		\end{verbatim}

		{\newpage \textbf{Тестовые данные}\par

		\begin{tikzpicture}
			\draw (0,0)--(5,0)--(5,-3)--(0,-3)--cycle;
			\draw (0,-1)--(5,-1);
			\draw (0,-2)--(5,-2);
			\draw (2.8,0)--(2.8,-3);
			\node at (1.2,-0.5) {\textbf{Ввод}};
			\node at (4,-0.5) {\textbf{Вывод}};
			\node at (1.2, -1.5) {3, 58, 12};
			\node at (1.2, -2.5) {30, 30, 30};
			\node at (4, -1.5) {0.069254};
			\node at (4, -2.5) {0.532201};
		\end{tikzpicture}}
		\newpage
	\end{flushleft}

	\begin{flushleft}
		{\textbf{Задание 2}}\par
		{\textbf{Tема 2. Разветвляющиеся алгоритмы}}\par
		{Заданы три числа: a, b, c. Определить, могут ли они быть сторонами
		треугольника, и если да, то определить его тип:
		равносторонний,	равнобедренный, разносторонний.}\par
		{\textbf{Математическое обоснование и словесное описание}\par
		Если стороны треугольника соответствуют условиям $a+b>c$ и $a+c>b$ и $b+c>a$,
		то он существует. Если $a=b$ или $b=c$ или $a=c$, то он равнобедренный.
		Если $a=b$ и $b=c$ и $a=c$ то он равносторонний. Если треугольник существует и
		невыполняется ни одно из условий, то он разносторонний.}\par

		{\textbf{Блок-схема}\par 
		\begin{center}
			\begin{tikzpicture}
				\filldraw [fill=black!5, very thick] ellipse (1.5 and 0.5);
				\node at (0, 0) {Начало};
				\draw [thick, ->] (0,-0.5) -- (0, -1);
				\draw (-2,-1)--(2,-1)--(2.5,-1.7)--(-1.5,-1.7)--cycle;
				\node at (0.2,-1.4) {ввод a, b, c};
				\draw [thick, ->] (0,-1.7) -- (0, -2.3);
				\draw (0,-2.3)--(3,-3.3)--(0,-4.3)--(-3,-3.3)--cycle;
				\node at (0,-3.1) {a+b$>$c and a+c$>$b};
				\node at (0,-3.6) {and b+c$>$a};
				\node at (3,-3) {+};
				\draw [thick, ->](-3,-3.3)--(-4,-3.3)--(-4,-4.3);
				\draw (-7,-4.3)--(-1.5,-4.3)--(-1.2,-5.3)--(-6.7,-5.3)--cycle;
				\node at (-4, -4.8) {Вывод Triangle doesn't exist.};
				\draw (-4,-5.3)--(-4,-9.2)--(0,-9.2);
				\draw (3,-3.3)--(7,-3.3);
				\draw [thick, ->] (7,-3.3)--(7,-3.7);
				\draw (7,-3.7)--(9,-4.5)--(7,-5.3)--(5,-4.5)--cycle;
				\node at (7,-4.3) {a=b or a=c};
				\node at (7,-4.8) {or b=c};
				\node at (5,-4) {+};
				\draw (9,-4.5)--(10,-4.5)--(10,-5.5);
				\draw (6.5,-5.5)--(10.5,-5.5)--(11,-6.5)--(7,-6.5)--cycle;
				\node at (9, -6) {вывод Scalene triangle.};
				\draw (10, -6.5)--(10, -8.9)--(7,-8.9);
				\draw [thick, ->] (5,-4.5)--(3,-4.5)--(3,-5);
				\draw (3,-5)--(5,-5.9)--(3,-6.9)--(1,-5.9)--cycle;
				\node at (3, -5.7) {a=b and a=c};
				\node at (3, -6.2) {and b=c};
				\node at (1,-5.5) {+};
				\draw [thick, ->] (1,-5.9)--(0,-5.9)--(0,-7);
				\draw (2.4,-7)--(-2.9,-7)--(-2.4, -8)--(2.9,-8)--cycle;
				\node at (0,-7.5) {Вывод Equilateral triangle.};
				\draw (0,-8)--(0,-8.5)--(3,-8.5);
				\draw [thick, ->] (5,-5.9)--(6,-5.9)--(6,-7);
				\draw (4,-7)--(8.5,-7)--(8.9, -8)--(4.3,-8)--cycle;
				\node at (6.5, -7.5) {вывод Isosceles triangle.};
				\draw [thick, ->] (6,-8)--(6, -8.5)--(3,-8.5)--(3, -8.9)--(7, -8.9)--(7, -9.2)--(0,-9.2)--(0,-9.5);
				\filldraw [fill=black!5, very thick](0,-10) ellipse (1.5 and 0.5);
				\node at (0, -10) {Конец};
			\end{tikzpicture}\par
		\end{center}}
		\newpage
		{\textbf{Код программы}}\par
		\begin{verbatim}
			void zadanie2(int a, int b, int c) {
    			if (a + b > c && a + c > b && b + c > a)
        			if (a == b || a == c || b == c)
           				 if (a == b && a == c && b == c)
                			printf("Equilateral triangle.\n");
           				 else
               				printf( "Isosceles triangle.\n");
       				 else
           				printf( "Scalene triangle.\n");
   				 else
        			printf( "Triangle doesn't exist.\n");
			}
		\end{verbatim}

		{\textbf{Тестовые данные}\par

		\begin{tikzpicture}
			\draw (0,0)--(7,0)--(7,-3)--(0,-3)--cycle;
			\draw (0,-1)--(7,-1);
			\draw (0,-2)--(7,-2);
			\draw (2.8,0)--(2.8,-3);
			\node at (1.2,-0.5) {\textbf{Ввод}};
			\node at (4,-0.5) {\textbf{Вывод}};
			\node at (1.2, -1.5) {30,30,30};
			\node at (1.2, -2.5) {30,30,20};
			\node at (5, -1.5) {Equilateral triangle.};
			\node at (5, -2.5) {Isosceles triangle.};
		\end{tikzpicture}}

	\end{flushleft}


	\begin{flushleft}
		\newpage
		{\textbf{Задание 3}}\par
		{\textbf{Tема 3. Циклические и итерационные алгоритмы}}\par
		{Численно убедиться, является ли заданная функция $y=f(x)$
		чётной или нёчетной на заданном отрезке $-a\leq x \leq a$. Учесть погрешность
		вычислений и возможные точки разрыва функции.}\par
		{\textbf{Математическое обоснование и словесное описание}\par
		Нечётными и чётными называются функции, обладающие симметрией
		относительно изменения знака аргумента.
		Если функция нечетная, то выполняется равенство:
		$f(x)+f(-x)=0$. Если функция четная, то выполняется равенство:
		$f(x)-f(-x)=0$. Пройдем от $-a$ до $a$ с шагом $\Delta =0.1$ и
		определим четность или нечетность функции.}\par

		{\textbf{\newpage Блок-схема}\par 
			\begin{flushleft}
				\begin{tikzpicture}
					\filldraw [fill=black!5, very thick] (1, 0) ellipse (3 and 0.5);
					\node at (1, 0) {odd(x, f())};
					\draw [thick, ->] (1,-0.5)--(1,-1);
					\draw (-1,-1)--(3,-1)--(3,-2)--(-1,-2)--cycle;
					\node at (1,-1.5) {res:=f(x)+f(-x)};
					\draw [thick, ->] (1,-2)--(1,-2.5);
					\filldraw [fill=black!5, very thick] (1, -3) ellipse (3 and 0.5);
					\node at (1, -3) {Возврат res};

					\filldraw [fill=black!5, very thick] (10, 0) ellipse (3 and 0.5);
					\node at (10, 0) {even(x, f())};
					\draw [thick, ->] (10,-0.5)--(10,-1);
					\draw (8,-1)--(12,-1)--(12,-2)--(8,-2)--cycle;
					\node at (10,-1.5) {res:=f(x)-f(-x)};
					\draw [thick, ->] (10,-2)--(10,-2.5);
					\filldraw [fill=black!5, very thick] (10, -3) ellipse (3 and 0.5);
					\node at (10, -3) {Возврат res};

					\filldraw [fill=black!5, very thick] (5.5, -4) ellipse (1.5 and 0.5);
					\node at (5.5, -4) {Начало};
					\draw [thick, ->] (5.5,-4.5)--(5.5,-5);
					\draw (3.5,-5)--(7,-5)--(7.5,-6)--(4,-6)--cycle;
					\node at (5.5, -5.5) {Ввод f(), a};
					\draw [thick, ->] (5.5,-6)--(5.5,-6.5);
					\draw (3.5,-6.5)--(7.5,-6.5)--(7.5,-8.5)--(3.5,-8.5)--cycle;
					\node at (5.5,-7) {x:=-a, step:=0.1,};
					\node at (5.5,-7.5) {epsilon:=1E-10,};
					\node at (5.5,-8) {isOdd:=1, isEven:=1};
					\draw [thick, ->] (5.5,-8.5)--(5.5,-9);
					\draw (5.5,-9)--(6.5,-10)--(5.5,-11)--(4.5,-10)--cycle;
					\node at (5.5, -10) {x<=a};
					\draw [thick, ->] (4.5,-10)--(0,-10)--(0,-10.5);
					\draw (0,-10.5)--(2,-11.5)--(0,-12.5)--(-2,-11.5)--cycle;
					\node at (0, -11.5) {f(x)$>$epsilon};
					\draw (-2,-11.5)--(-2,-18)--(2.5,-18)--(2.5,-17.5);
					\draw [thick, ->] (2,-11.5)--(2.5,-11.5)--(2.5,-12);
					\draw (2.5,-12)--(4.5,-13)--(2.5,-14)--(0.5,-13)--cycle;
					\node at (2.6, -13) {|odd(x,f)|$<$epsilon};
					\draw [thick, ->] (0.5,-13)--(0.5, -13.6);
					\draw (-0.5,-13.6)--(1.5,-13.6)--(1.5,-14.6)--(-0.5,-14.6)--cycle;
					\node at (0.5, -14.1) {isEven:=0};
					\draw (0.4, -14.6)--(0.4,-17.5)--(5,-17.5)--(5,-17);
					\draw [thick, ->] (4.5,-13)--(5,-13)--(5,-13.5);
					\draw (5,-13.5)--(7,-14.5)--(5,-15.5)--(3,-14.5)--cycle;
					\node at (5, -14.5) {|even(x,f)|$<$epsilon};
					\draw [thick, ->] (7,-14.5)--(7.5,-14.5)--(7.5,-15.5);
					\draw (6.5,-15.5)--(8.5,-15.5)--(8.5,-16.5)--(6.5,-16.5)--cycle;
					\node at (7.5,-16) {isOdd:=0};
					\draw [thick, ->] (3,-14.5)--(2.5,-14.5)--(2.5,-15);
					\draw (0.5,-15)--(3.9,-15)--(3.9,-16.5)--(0.5,-16.5)--cycle;
					\node at (2, -15.4) {isEven:=0};
					\node at (2,-16) {isOdd:=0};
					\draw (2.5, -16.5)--(2.5,-17)--(7.5,-17)--(7.5,-16.5);
					\draw [thick, ->] (0.5, -18)--(0.5,-18.5);
					\draw (-1.5, -18.5)--(2.5,-18.5)--(2.5,-19.5)--(-1.5,-19.5)--cycle;
					\node at (0.5, -19) {x:=x+steep};
					\draw [thick, ->](0.5, -19.5)--(0.5, -20)--(-2.5, -20)--(-2.5,-9)--(5.5,-9);
					\draw [thick, ->] (6.5, -10)--(8.6,-10)--(8.6,-18)--(5.5,-18)--(5.5,-18.5);
					\draw (5.5,-18.5)--(6.5,-19)--(5.5,-19.5)--(4.5,-19)--cycle;
					\node at (5.5, -19) {isOdd};
					\draw (4.5,-19)--(3.5,-19)--(3.5,-19.5);
					\draw (3,-19.5)--(5,-19.5)--(4.5,-20)--(2.5,-20)--cycle;
					\node at (3.6, -19.8) {odd};
					\draw [thick, ->] (6.5,-19)--(7,-19)--(7,-19.5);
					\draw (7,-19.5)--(8,-20)--(7,-20.5)--(6,-20)--cycle;
					\node at (7, -20) {isEven};
					\draw [thick, ->] (6,-20)--(5.5,-20)--(5.5,-20.5);
					\draw (5,-20.5)--(6.5,-20.5)--(6,-21)--(4.5,-21)--cycle;
					\node at (5.5, -20.8) {even};
					\draw [thick, ->] (8,-20)--(8.5,-20)--(8.5,-20.5);
					\draw (7.5,-20.5)--(12,-20.5)--(11.5,-21)--(7,-21)--cycle;
					\node at (9.5, -20.8) {neither odd nor even};
					\draw (8.5,-21)--(8.5,-21.5)--(5.5,-21.5)--(5.5,-21);
					\draw (7,-21.5)--(7,-22)--(3.5,-22)--(3.5,-20);
					\draw [thick, ->](5.5,-22)--(5.5,-23);
					\filldraw [fill=black!5, very thick] (5.5, -23) ellipse (1.5 and 0.5);
					\node at (5.5, -23) {Конец};
					
				
				\end{tikzpicture}
			\end{flushleft}
		}
	\newpage
		{\textbf{Код программы}}\par
		\begin{verbatim}
			/*  проверка на нечетность. Возвращает ~0 если нечетная */
			double odd(double x, double (*f)(double)){
    			return f(x) + f(-x);
			}
			/*  проверка на четность. Возвращает ~0 если четная */
			double even(double x, double (*f)(double)){
    			return f(x) - f(-x);
			}
			double fabs(double a){
   				if(a < 0)
        			return -a;
   			return a;
			}
			/*  общая проверка, вывод результата    */
			void zadanie3(double(*f)(double), double a){
				double x = -a, step = 0.1, epsilon = 1E-10;
				int isOdd = 1, isEven = 1;

				while(x <= a){
					if(f(x) > epsilon){
						if(fabs(odd(x,f)) < epsilon)
							isEven *= 0;
						else if(fabs(even(x,f)) < epsilon)
							isOdd *= 0;
						else{
							isEven *= 0;
							isOdd *= 0;
						}
					}
					x += step;
				}

				if(isOdd)
					printf("odd");
				else if (isEven)
					printf("even");
				else
					printf("neither odd nor even");
			}
		\end{verbatim}

		{\textbf{Тестовые данные}\par

		\begin{tikzpicture}
			\draw (0,0)--(7,0)--(7,-3)--(0,-3)--cycle;
			\draw (0,-1)--(7,-1);
			\draw (0,-2)--(7,-2);
			\draw (2.8,0)--(2.8,-3);
			\node at (1.2,-0.5) {\textbf{Ввод}};
			\node at (4,-0.5) {\textbf{Вывод}};
			\node at (1.2, -1.5) {sin, 50};
			\node at (1.2, -2.5) {cos, 5};
			\node at (5, -1.5) {odd};
			\node at (5, -2.5) {even};
		\end{tikzpicture}}

	\end{flushleft}

	\begin{flushleft}
		\newpage
		{\textbf{Задание 4}}\par
		{\textbf{Tема 4. Простейшие операции над массивами}}\par
		{В массиве $C(n)$ подсчитать количество отрицательных и сумму
		положительных элементов.}\par
		{\textbf{Математическое обоснование и словесное описание}\par
		Пройдем последовательно по массиву $C(n)$ и если элемент $\geq 0$
		то будем прибавлять к счетчику $sum\_positive$, если нет, то прибавим 1 к
		$quantity\_negative$.
		}\par
	\end{flushleft}

	{\textbf{Блок-схема}\par 
			\begin{center}
				\begin{tikzpicture}
					\filldraw [fill=black!5, very thick] (0, 0) ellipse (1.5 and 0.5);
					\node at (0, 0) {Начало};
					\draw [thick, ->] (0,-0.5)--(0,-1);
					\draw (-2,-1)--(2,-1)--(2,-2.5)--(-2,-2.5)--cycle;
					\node at (0, -1.5) {sum\_positive:=0};
					\node at (0, -2) {quantity\_negative:=0};
					\draw [thick, ->] (0,-2.5)--(0,-3);
					\draw (-2,-3)--(2,-3)--(2.5,-3.5)--(2,-4)--(-2,-4)--(-2.5,-3.5)--cycle;
					\node at (0, -3.5) {0 ... n};
					\draw [thick, ->] (0, -4)--(0, -4.5);
					\draw (0, -4.5)--(1, -5.5)--(0, -6.5)--(-1, -5.5)--cycle;
					\node at (0, -5.5) {c[i]<0};
					\draw [thick, ->] (1,-5.5)--(2,-5.5)--(2,-7);
					\draw (0.1,-7)--(4,-7)--(4,-8)--(0.1,-8)--cycle;
					\node at (2,-7.3) {\small quantity\_negative:=};
					\node at (2,-7.7) {\small quantity\_negative+1};
					\draw [thick, ->] (-1,-5.5)--(-2,-5.5)--(-2,-7);
					\draw (-4,-7)--(-0.1,-7)--(-0.1,-8)--(-4,-8)--cycle;
					\node at (-2, -7.3) {\small sum\_positive:=};
					\node at (-2, -7.7) {\small sum\_positive+c[i]};
					\draw (-2, -8)--(-2,-8.5)--(2,-8.5)--(2,-8);
					\draw [->] (0,-8.5)--(0,-9)--(-4.5,-9)--(-4.5,-3.5)--(-2.5,-3.5);
					\draw [->] (2.5, -3.5)--(4.5, -3.5)--(4.5,-9.5)--(0,-9.5)--(0,-10);
					\draw (-2.5,-10)--(2,-10)--(2.5,-11.5)--(-2,-11.5)--cycle;
					\node at (0, -10.5) {Вывод sum\_positive};
					\node at (0, -11) {quantity\_negative};
					\draw [->] (0,-11.5)--(0,-12);
					\filldraw [fill=black!5, very thick] (0, -12.5) ellipse (1.5 and 0.5);
					\node at (0, -12.5) {Конец};

				\end{tikzpicture}
			\end{center}}

			\newpage
		{\textbf{Код программы}}\par
		\begin{verbatim}
			void zadanie4(int *c, int n) {
				int sum_positive = 0;
				int quantity_negative = 0;
				for(int i = 0; i < n; i++) {
					if(c[i] < 0)
						quantity_negative++;
					else
						sum_positive += c[i];
				}

				printf("\n%d - sum positive elements\n", sum_positive);
				printf("%d - quantity negative elements\n", quantity_negative);
			}
		\end{verbatim}

		{\textbf{Тестовые данные}\par

		\begin{tikzpicture}
			\draw (0,0)--(15,0)--(15,-3)--(0,-3)--cycle;
			\draw (0,-1)--(15,-1);
			\draw (0,-2)--(15,-2);
			\draw (5,0)--(5,-3);
			\node at (3,-0.5) {\textbf{Ввод}};
			\node at (9,-0.5) {\textbf{Вывод}};
			\node at (2.4, -1.5) {\{1,2,3,4,4,0,-1,-2,13\}, 9};
			\node at (2.5, -2.5) {\{-1,-2,-3,-4,-5,-6,7,8,9\}, 9};
			\node at (9, -1.5) {27 - sum positive 2 - quantity negative};
			\node at (9, -2.5) {24 - sum positive 6 - quantity negative};
		\end{tikzpicture}}


		\begin{flushleft}
			\newpage
			{\textbf{Задание 5}}\par
			{\textbf{Tема 5. Векторы и матрицы}}\par
			{Строки матрицы $A(m,n)$ заполнены не полностью: в массиве
			$L(m)$ указано количество элементов в каждой строке. Переслать элементы
			 матрицы построчно в начало одномерного массива $T(m\cdot n)$ и
			подсчитать их количество.}\par
			{\textbf{Математическое обоснование и словесное описание}\par
			Пройдем по столбцам матрицы $A(m,n)$ и по строкам, в соответствии
			с числом, хранящимся в массиве $L(m)$. Будем добавлять в массив $T(m\cdot n)$
			элементы, попутно считая их в счетчике $size\_T$.
			}\par

		\end{flushleft}

		{\textbf{Блок-схема}\par 
			\begin{center}
				\begin{tikzpicture}
					\filldraw [fill=black!5, very thick] (0, 0) ellipse (1.5 and 0.5);
					\node at (0, 0) {Начало};
					\draw [->] (0,-0.5)--(0,-1);
					\draw (-2,-1)--(2,-1)--(2.5,-2.5)--(-1.5,-2.5)--cycle;
					\node at (0, -1.5) {Ввод A(m,n), L(m),};
					\node at (0, -2) {T(m$*$n), m, n};
					\draw [->] (0,-2.5)--(0,-3);
					\draw (-1,-3)--(1,-3)--(1,-4)--(-1,-4)--cycle;
					\node at (0,-3.5) {size\_T:=0};
					\draw [->] (0,-4)--(0,-4.5);
					\draw (-1,-4.5)--(1,-4.5)--(1.5,-5)--(1,-5.5)--(-1,-5.5)--(-1.5,-5)--cycle;
					\node at (0, -5) {i:=0 ... m};
					\node at (0.2,-5.7) {+};
					\draw [->] (0, -5.5)--(0,-6);
					\draw (-1,-6)--(1,-6)--(1.5,-6.5)--(1,-7)--(-1,-7)--(-1.5,-6.5)--cycle;
					\node at (0,-6.5) {j:=0 ... $L_i$};
					\node at (0.2,-7.2) {+};
					\draw [->] (0,-7)--(0,-7.5);
					\draw (-2,-7.5)--(2,-7.5)--(2,-9.5)--(-2,-9.5)--cycle;
					\node at (0,-8) {$T_{size\_T}$:=$A_{i,j}$};
					\node at (0,-8.5) {size\_T:=};
					\node at (0,-9) {size\_T+1};
					\draw [->] (0, -9.5)--(0,-10)--(-2.5,-10)--(-2.5, -6.5)--(-1.5,-6.5);
					\draw [->] (1.5,-6.5)--(2.5,-6.5)--(2.5,-10.5)--(-3,-10.5)--(-3,-5)--(-1.5,-5);
					\draw [->] (1.5,-5)--(3,-5)--(3,-11)--(0,-11)--(0,-11.5);
					\draw (-2.5,-11.5)--(2,-11.5)--(2.5,-12.5)--(-2,-12.5)--cycle;
					\node at (0,-12) {Вывод size\_T};
					\draw [->] (0,-12.5)--(0,-13);
					\filldraw [fill=black!5, very thick] (0, -13.5) ellipse (1.5 and 0.5);
					\node at (0, -13.5) {Конец};
				\end{tikzpicture}
			\end{center}}


			\newpage
		{\textbf{Код программы}}\par
		\begin{verbatim}
			int zadanie5(int **A, int *L, int *T, int m, int n) {
				int size_T = 0;
				for(int i = 0; i < m; i++) {
					for(int j = 0; j < L[i]; j++) {
						T[size_T++] = A[i][j];
					}
				}

				return size_T;
			}
		\end{verbatim}

		{\textbf{Тестовые данные}\par

		\begin{tikzpicture}
			\draw (0,0)--(17,0)--(17,-4)--(0,-4)--cycle;
			\draw (0,-1)--(17,-1);
			\draw (0,-2.5)--(17,-2.5);
			\draw (8,0)--(8,-4);
			\node at (1,-0.5) {\textbf{Ввод}};
			\node at (9,-0.5) {\textbf{Вывод}};
			\node at (2.6,-1.5) {A=\{\{1,2,3\},\{4,5,6\},\{7,8,9\}\}};
			\node at (2.2,-2) {L=\{1,1,2\}, T=\{\}, 3, 3};
			\node at (9.5, -1.5) {T=\{1,4,7,8\}, 4};
			\node at (2.6,-3) {A=\{\{1,1,1\},\{2,2,2\},\{3,3,3\}\}};
			\node at (2.2,-3.5) {L=\{2,1,0\}, T=\{\}, 3, 3};
			\node at (9.5, -3) {T=\{1,1,2\}, 3};
		\end{tikzpicture}}

		\newpage
		\begin{flushleft}
			\newpage
			{\textbf{Задание 6}}\par
			{\textbf{Tема 6. Линейный поиск}}\par
			{Седловой точкой в матрице называется элемент, являющийся
			одновременно наибольшим в столбце и наименьшим в строке. Седловых 
			точек может быть несколько. В матрице $A(m,n)$ найти все седловые
			точки либо установить, что таких точек нет.}\par
			{\textbf{Математическое обоснование и словесное описание}\par
			Пройдем по столбцам матрицы $A(m,n)$. Пройдем по строкам, найдем минимальный элемент и
			его индекс. Пройдем по столбцу с минимальным индексом и найдем максимальный элемент.
			Если максимальный элемент будет равен минимальному, то выводим элемент
			и координаты, в которых он находится.
			}\par

		\end{flushleft}
		\newpage
		{\textbf{Блок-схема}\par 
			\begin{center}
				\begin{tikzpicture}
					\filldraw [fill=black!5, very thick] (0, 0) ellipse (1.5 and 0.5);
					\node at (0, 0) {Начало};
					\draw [->] (0,-0.5)--(0,-1);
					\draw (-2,-1)--(2,-1)--(2.5,-2)--(-1.5,-2)--cycle;
					\node at (0, -1.5) {Ввод A(m,n), m, n};
					\draw [->] (0,-2)--(0,-2.5);
					\draw (-1,-2.5)--(1,-2.5)--(1.5,-3)--(1,-3.5)--(-1,-3.5)--(-1.5,-3)--cycle;
					\node at (0,-3) {i:=0...m};
					\node at (0.2,-3.7) {+};
					\draw [->] (0,-3.5)--(0,-4);
					\draw (-2,-4)--(2,-4)--(2,-5.5)--(-2,-5.5)--cycle;
					\node at (0,-4.5) {min\_element:=$A_{i,0}$};
					\node at (0,-5)	{min\_index:=0};
					\draw [->] (0, -5.5)--(0,-6);
					\draw (-1,-6)--(1,-6)--(1.5,-6.5)--(1,-7)--(-1,-7)--(-1.5,-6.5)--cycle;
					\node at (0, -6.5) {j:=1...n};
					\node at (0.2,-7.3) {+};
					\draw [->] (0,-7)--(0,-7.5);
					\draw (0,-7.5)--(2,-8.5)--(0,-9.5)--(-2,-8.5)--cycle;
					\node at (2.1, -8.2) {+};
					\node at (0,-8.5) {\small min\_element$>$$A_{i,j}$};
					\draw [->] (2,-8.5)--(3,-8.5)--(3,-9.5);
					\draw (1,-9.5)--(5,-9.5)--(5,-11)--(1,-11)--cycle;
					\node at (3, -10) {min\_element:=$A_{i,j}$};
					\node at (3, -10.5) {min\_index:=j};
					\draw [->] (3,-11)--(3,-11.5)--(-2.5,-11.5)--(-2.5,-6.5)--(-1.5,-6.5);
					\draw (-2,-8.5)--(-2.5,-8.5);
					\draw [->] (1.5,-6.5)--(5.5,-6.5)--(5.5,-12)--(0,-12)--(0,-12.5);
					\draw (-2.5,-12.5)--(2.5,-12.5)--(2.5,-13.5)--(-2.5,-13.5)--cycle;
					\node at (0,-13) {max\_element:=$A_{0,min\_index}$};
					\draw [->](0,-13.5)--(0,-14);
					\draw (-1,-14)--(1,-14)--(1.5,-14.5)--(1,-15)--(-1,-15)--(-1.5,-14.5)--cycle;
					\node at (0, -14.5) {j:=1...n};
					\node at (0.2,-15.3) {+};
					\draw [->] (0,-15)--(0,-15.5);
					\draw (0,-15.5)--(3,-16.5)--(0,-17.5)--(-3,-16.5)--cycle;
					\node at (0,-16.5) {\small $A_{j,min\_index}>$max\_element};
					\node at (2.5,-16) {+};
					\draw [->] (3,-16.5)--(3.5,-16.5)--(3.5,-17.5);
					\draw (1,-17.5)--(6,-17.5)--(6,-18.5)--(1,-18.5)--cycle;
					\node at (3.5,-18) {max\_element=$A_{j,min\_index}$};
					\draw (3.5,-18.5)--(3.5,-19)--(-3.5,-19)--(-3.5,-16.5)--(-3,-16.5);
					\draw [->] (0, -19)--(0,-19.5)--(-4,-19.5)--(-4,-14.5)--(-1.5,-14.5);
					\draw [->] (1.5,-14.5)--(6.5,-14.5)--(6.5,-20)--(0,-20)--(0,-20.5);
					\filldraw [fill=black!5, very thick] (0, -21) ellipse (0.5 and 0.5);
					\node at (0,-21) {1};
					\filldraw [fill=black!5, very thick] (-4.5, -21) ellipse (0.5 and 0.5);
					\node at (-4.5,-21) {2};
					\draw [->] (-4.5,-20.5)--(-4.5,-3)--(-1.5,-3);
					\filldraw [fill=black!5, very thick] (7, -21) ellipse (0.5 and 0.5);
					\node at (7,-21) {3};
					\draw (1.5,-3)--(7,-3)--(7,-20.5);
					
				\end{tikzpicture}
				\newpage
				\begin{tikzpicture}
					\filldraw [fill=black!5, very thick] (0, 0) ellipse (0.5 and 0.5);
					\node at (0,0) {1};
					\draw [->] (0,-0.5)--(0,-1);
					\draw (0,-1)--(2,-2)--(0,-3)--(-2,-2)--cycle;
					\node at (0,-1.9) {max\_element:=};
					\node at (0,-2.3) {$A_{j,min\_index}$};
					\node at (2,-1.5) {+};
					\draw [->] (2,-2)--(2.5,-2)--(2.5,-3);
					\draw (0,-3)--(5,-3)--(5,-4)--(0,-4)--cycle;
					\node at (2.5,-3.5) {max\_element:=$A_{j,min_index}$};
					\draw (2.5,-4)--(2.5,-4.5)--(-2.5,-4.5)--(-2.5,-2)--(-2,-2);
					\draw [->] (0,-4.5)--(0,-5)--(-3,-5)--(-3,-0.5);
					\filldraw [fill=black!5, very thick] (-3, 0) ellipse (0.5 and 0.5);
					\node at (-3,0) {2};
					\filldraw [fill=black!5, very thick] (6, 0) ellipse (0.5 and 0.5);
					\node at (6,0) {3};
					\draw [->](6,-0.5)--(6,-5.5)--(0,-5.5)--(0,-6);
					\filldraw [fill=black!5, very thick] (0,-6.5) ellipse (1.5 and 0.5);
					\node at (0,-6.5) {Конец};
				\end{tikzpicture}
			\end{center}}
			\newpage
			{\textbf{Код программы}}\par
		\begin{verbatim}
			void zadanie6(int **A, int m, int n) {
				for(int i = 0; i < m; i++) {
					int min_element = A[i][0];
					int min_index = 0;

					for(int j = 1; j < n; j++)
						if(min_element > A[i][j])
							min_element = A[i][j], min_index = j;

					int max_element = A[0][min_index];
					for(int j = 1; j < m; j++)
						if(A[j][min_index] > max_element)
							max_element = A[j][min_index];

					if(max_element == min_element)
						printf("\n(%d) (%d, %d) is saddle point\n", min_element, i, min_index);
				}
			}
		\end{verbatim}

		{\textbf{Тестовые данные}\par

		\begin{tikzpicture}
			\draw (0,0)--(17,0)--(17,-4)--(0,-4)--cycle;
			\draw (0,-1)--(17,-1);
			\draw (0,-2.5)--(17,-2.5);
			\draw (8,0)--(8,-4);
			\node at (1,-0.5) {\textbf{Ввод}};
			\node at (9,-0.5) {\textbf{Вывод}};
			\node at (3.8,-1.5) {A=\{\{1,2,3\},\{4,5,6\},\{7,8,9\},\{10,11,12\}\},};
			\node at (1,-2.2) {4, 3};
			\node at (10.5, -1.5) {(10) (3, 0) is saddle point};
			\node at (3.5,-3) {A=\{\{1,2,3\},\{4,5,6\},\{7,8,9\},\{1,1,1\}\},};
			\node at (1,-3.5) {4, 3};
			\node at (10.5, -3) {(7) (2, 0) is saddle point};
		\end{tikzpicture}}

\end{document}
