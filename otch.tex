\documentclass[12pt, a4paper]{article}

\usepackage[english, russian]{babel}
\usepackage[T2A]{fontenc}
\usepackage[utf8]{inputenc}

\usepackage{amsmath}
\usepackage{tikz}

\usepackage[left=3cm,right=1.5cm,top=2cm,bottom=2cm,bindingoffset=0cm]{geometry}

\pagestyle{empty}
\linespread{1.5}






\begin{document}
	
	\begin{center}
	 	{\large МИНИСТЕРСТВО НАУКИ И ВЫСШЕГО ОБРАЗОВАНИЯ РОССИЙСКОЙ ФЕДЕРАЦИИ} 

		{\large ФЕДЕРАЛЬНОЕ ГОССУДАРСТВЕННОЕ БЮДЖЕТНОЕ ОБРАЗОВАТЕЛЬНОЕ УЧРЕЖДЕНИЕ ВЫСШЕГО ОБРАЗОВАНИЯ}
		\begin{center}
			{\large \textbf{БЕЛГОРОДСКИЙ ГОССУДАРСТВЕННЫЙ ТЕХНОЛОГИЧЕСКИЙ УНИВЕРСИТЕТ \mbox{им. В. Г. Шухова} (БГТУ им. В. Г. Шухова)}}
		\end{center}	
	\end{center}
	
	\begin{center}
		{\large ИНСТИТУТ ИНФОРМАЦИОННЫХ ТЕХНОЛОГИЙ И УПРАВЛЯЮЩИХ СИСТЕМ}
	\end{center}
	\begin{center}
		{\large Кафедра программного обеспечения вычислительной техники и автоматизированных систем}
	\end{center}
	\begin{center}
		\vspace{1cm}
		{\textbf {\Large Отчет}}

		{\Large По учебно-ознакомительной практике}
	
	\end{center}

	\begin{flushright}
		\vspace{1cm}
		{\large Выполнил: студент группы КБ-232}

		{\large Башков Михаил Антонович}

		\vspace{0.5cm}
		\begin{tikzpicture}
			\draw[black, thick] (0,0) -- (5,0);
		\end{tikzpicture}

		{(подпись студента)}

		\vspace{0.5cm}
		{\large Проверил: асистент \mbox{Новожен Н.В.}}
		\vspace{0.5cm}

		\begin{tikzpicture}
			\draw[black, thick] (0,0) -- (5,0);
		\end{tikzpicture}

		{(подпись руководителя практики)}

		\vspace{0.3cm}
		{\large Оценка:}
		\begin{tikzpicture}
			\draw[black, thick] (0,0) -- (3,0);
		\end{tikzpicture}

	\end{flushright}
	
	\begin{center}
		\vspace{3cm}
		{\large Белгород 2024 г.}
	\end{center}
	




	\begin{center}
		\newpage
		{\Large \textbf{Оглавление}}
	\end{center}
	\begin{flushleft}
		{\large \textbf{Компьютерная практика}}
		
		{1. Tема 1. Линейные алгоритмы} \par
		{2. Tема 2. Разветвляющиеся алгоритмы} \par
		{3. Tема 3. Циклические и итерационные алгоритмы} \par
		{4. Tема 4. Простейшие операции над массивами} \par
		{5. Tема 5. Векторы и матрицы} \par
		{6. Tема 6. Линейный поиск} \par
		{7. Tема 7. Арифметика} \par
		{8. Tема 8. Геометрия и теория множеств} \par
		{9. Tема 9. Линейная алгебра и сжатие информации} \par
		{10. Tема 10. Алгоритмы обработки символьной информации} \par
		{11. Tема 11. Аналитическая геометрия} \par
		{12. Tема 12. Кривые второго порядка на плоскости} \par
		{13. Tема 13. Графическое решение систем уравнений} \par
		{14. Tема 14. Плоскость в трехмерном пространстве} \par
		{15. Tема 15. Поверхность второго порядка в трехмерном пространстве} \par
	\end{flushleft}



	\begin{flushleft}
		\newpage
		{\Large \textbf{Задания к работе}} \par
		{Tема 1. \par Угол $\alpha$ задан в градусах, минутах и секундах. Найти его величину в радианах (с максимально возможной точностью).} \par
		{Tема 2. \par Заданы три числа: a, b, c. Определить, могут ли они быть сторонами треугольника,
		 и если да, то определить его тип: равносторонний,	равнобедренный, разносторонний.} \par
		{Tема 3. \par Численно убедиться, является ли заданная функция $y=f(x)$
		чётной или нёчетной на заданном отрезке $-a\leq x \leq a$. Учесть погрешность
		вычислений и возможные точки разрыва функции.} \par
		{Tема 4. \par В массиве $C{n}$ подсчитать количество отрицательных и сумму
		положительных элементов.} \par
		{Tема 5. \par Строки матрицы $A(m,n)$ заполнены не полностью: в массиве
		$L(m)$указано количество элементов в каждой строке. Переслать элементы
		 матрицы построчно в начало одномерного массива $T(m\cdot n)$ и
		подсчитать их количество.} \par
		{Tема 6. \par Седловой точкой в матрице называется элемент, являющийся
		одновременно наибольшим в столбце и наименьшим в строке. Седловых 
		точек может быть несколько. В матрице $A(m,n)$ найти все седловые
		точки либо установить, что таких точек нет.} \par
		{Tема 7. \par Натуральное число в $p$-ичной системе счисления задано своими
		цифрами, хранящимися в массиве $K(n)$. Проверить корректность такого 
		представления и перевести число в $q$-ичную систему (возможно,
		число слишком велико, чтобы получить его внутреннее представление;
		кроме того, $p\leq 10$, $q\leq 10$).} \par
		{Tема 8. \par Заяц, хаотично прыгая, оставил след в виде замкнутой самопересекающейся
		ломаной, охватывающей территорию его владения (отрезки
		ломаной заданы длиной прыжка и его направлением по азимуту).
		Найти площадь минимального по площади выпуклого многоугольника,
		описанного вокруг этой территории.} \par
		\newpage
		{Tема 9. \par Выполнить операцию транспонирования прямоугольной матрицы
		$A(m,n)$, $m\neq n$, не выделяя дополнительного массива для хранения
		результата. Матрицу представить в виде одномерного массива.} \par
		{Tема 10. \par Текст записан одной длинной строкой. Признаком красной строки
		служит символ \$. Переформатировать текст в 60-символьные строки,
		формируя абзацы.} \par
		{Tема 11. \par Построить прямую параллельную оси абсцисс $(Ox)$ и пересекающую
		ось ординат $(Oy)$ в точке $A(0,2)$ в диапазоне $x\in [-3;3]$ с шагом $\Delta =0.5$.} \par
		{Tема 12. \par Построить верхнюю часть параболы $y^2=x$ при $0\leq x \leq 4$ с шагом $\Delta =0.25$.} \par
		{Tема 13. \par 
		\begin{equation*} 
			\begin{cases} 
				y=\frac{2}{x}+2
				\\
				z=x^2+1
			\end{cases} 
		\end{equation*}
		\begin{center}в диапазоне $0.2\leq x\leq 3$, с шагом $\Delta 0.1$.\end{center}} \par
		{Tема 14. \par Построить плоскость, параллельную плоскости $Oxy$ и пересекающую ось $Oz$
		в точке $M(0,0,2)$, при $0\leq x\leq 6$ с шагом $\Delta =0.5$ и $0\leq y\leq 6$
		с шагом $\Delta =1$.} \par
		{Tема 15. \par Построить верхнюю часть эллипсоида, заданного уравнением 
		$\frac{x^2}{9}+\frac{y^2}{4}+z^2=1$, лежащую в диапазоне $-3\leq x\leq 3$, $-2\leq y\leq 2$ с 
		шагом $\Delta =0.5$ для обеих переменных.} \par
	\end{flushleft}

\end{document}
